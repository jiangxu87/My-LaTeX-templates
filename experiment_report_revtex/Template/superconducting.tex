\documentclass[aps, twocolumn, pra, superscriptaddress, amsmath, showpacs, tightenlines, letterpaper,showkeys] {revtex4-1}
\usepackage{epsfig,graphicx,times}
\usepackage{amstext}
\usepackage{amsmath}            %serve per le subequazioni
\usepackage{amssymb}            %serve per il simbolo "marchio registrato", \circledR
\usepackage{amsthm}
\usepackage{leftidx}
%\usepackage{fancyhdr}
\usepackage{mathrsfs}
\usepackage{graphicx}           %serve per le figure eps, ps etcyy
\usepackage{subfigure}
\usepackage{latexsym}
\usepackage{bm}
%\usepackage{epstopdf}
\usepackage{epsfig}
\usepackage{xeCJK}
\usepackage[colorlinks=true]{hyperref}

\usepackage{fancyhdr}
\begin{document}

\title{Michelson测折射率}

\author{陆思锐}
\email{lsr15@mails.tsinghua.edu.cn}
\affiliation{Department of Physics, Tsinghua University, Beijing 100084, China}

\date{October 8, 2016}


\begin{abstract}
  超导现象本实验利用杜瓦瓶和液氮已经铂电阻温度计对高温超导现象进行观察和测量.

\end{abstract}

%\pacs{03.67.Lx, 03.67.Mn, 74.20.Fg}
\keywords{高温超导; 转变温度; 电阻测量}

\maketitle

\pagenumbering{arabic}

%\tableofcontents

\pagestyle{fancy}
\lhead{\textsc{Number 14, Wednesday Evening}} 
\chead{\textsc{ General Physics Lab (3) }} 
\rhead{\today} 
\lfoot{} 
\cfoot{\thepage}
\rfoot{} 
\section{Introduction}\label{introduction}
\subsection{实验目的}



\subsection{实验原理}


\subsection{实验装置} 



\section{实验内容}



\subsection{注意事项}




\section{实验数据和分析}
\subsection{记录参数}

\section{实验误差分析}

\subsection{总体}





\section{Acknowledgements}

Thank Prof. A for his remarkable teaching

%感谢基础物理实验3的任课老师: 肖志刚老师, 关分海老师, 唐如麟老师以及顾问朱鹤年老师. 尤其感谢本实验的负责教师关分海。

\bibliographystyle{plain}
\nocite{scully1999quantum}
\bibliography{refs}

% \newpage
% \clearpage
\appendix

\section{思考题}
\begin{enumerate}
\item Why egg
Chicken!

\end{enumerate}


\end{document}
