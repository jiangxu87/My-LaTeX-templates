\documentclass[ignorenonframetext,]{beamer}


\setbeamertemplate{caption}[numbered]
\setbeamertemplate{caption label separator}{: }
\setbeamercolor{caption name}{fg=normal text.fg}
\beamertemplatenavigationsymbolsempty
\usepackage{lmodern}

\usepackage{beamerthemeTsinghua}

%\mode<presentation> {
%\usetheme{Rochester}
%\usecolortheme{seahorse}
%}

\usepackage{xeCJK}
\usepackage{amssymb,amsmath}
\usepackage{ifxetex,ifluatex}
\usepackage{fixltx2e} % provides \textsubscript


\newif\ifbibliography
\usepackage{natbib}
\bibliographystyle{plainnat}
\hypersetup{
            pdftitle={Codes for Simultaneous Transmission of Quantum and Classical Information},
            pdfauthor={Sirui Lu},
            pdfborder={0 0 0},
            breaklinks=true}
\urlstyle{same}  % don't use monospace font for urls





% Prevent slide breaks in the middle of a paragraph:
\widowpenalties 1 10000
\raggedbottom

\AtBeginPart{
  \let\insertpartnumber\relax
  \let\partname\relax
  \frame{\partpage}
}
\AtBeginSection{
  \ifbibliography
  \else
    \let\insertsectionnumber\relax
    \let\sectionname\relax
    \frame{\sectionpage}
  \fi
}
\AtBeginSubsection{
  \let\insertsubsectionnumber\relax
  \let\subsectionname\relax
  \frame{\subsectionpage}
}

\setlength{\parindent}{0pt}
\setlength{\parskip}{6pt plus 2pt minus 1pt}
\setlength{\emergencystretch}{3em}  % prevent overfull lines
\providecommand{\tightlist}{%
  \setlength{\itemsep}{0pt}\setlength{\parskip}{0pt}}
\setcounter{secnumdepth}{0}

\title{Codes for Simultaneous Transmission of Quantum and Classical Information}
\author{Sirui Lu}
\institute{ISIT 2017}
\date{June 28, 2017}

\begin{document}
\frame{\titlepage}

\begin{frame}
\frametitle{Outline} % Table of contents slide, comment this block out to remove it
\tableofcontents% Throughout your presentation, if you choose to use \section{} and \subsection{} commands, these will automatically be printed on this slide as an overview of your presentation
\end{frame}

\section{Introduction}\label{introduction}

\begin{frame}{Introduction}

\begin{itemize}
\tightlist
\item
  The simultaneous transmission of both quantum and classical
  information over a quantum channel was initially investigated in
  \citeyearpar{DeSh05} from an information theoretic point of view, and
  followed up by many others (see, e. g.
  \citeyearpar[\citet{hsieh2010entanglement},
  \citet{hsieh2010trading}]{yard2005simultaneous}).
\end{itemize}

\end{frame}

\section{Background and Notations}\label{background-and-notations}

\begin{frame}{Background and Notations}

Our discussion is based on the theory of stabilizer quantum codes and
its connection to classical error-correcting codes (see, e. g.,
\citet{CRSS98}). We use the following notations.

\end{frame}

\section{Results}\label{results}

\begin{frame}{Results (Code Search)}

We perform a search for \(\mathcal{C}=[\![n,k{:}m,d]\!]_2\) codes with
distance \(d\ge 3\).

\end{frame}

\section{Discussion}\label{discussion}

\begin{frame}{Discussion}

\begin{itemize}
\tightlist
\item
  We have characterized hybrid quantum codes for the simultaneous
  transmission of quantum and classical information in terms of
  generalized Knill-Laflamme conditions.
\end{itemize}

\end{frame}

\section{Conclusions}\label{conclusions}

\begin{frame}{Conclusions}

\begin{itemize}
\tightlist
\item
  We consider the characterization as well as the construction of
  quantum codes that allow to transmit both quantum and classical
  information, which we refer to as ``\textbf{hybrid codes}''.
\end{itemize}

\end{frame}

\renewcommand\refname{Reference}
\begin{frame}[allowframebreaks]{Reference}
\bibliographytrue
\bibliography{ref.bib}
\end{frame}


\begin{frame}
\begin{center}
\Huge{{Thank you!}}
\end{center}
\end{frame}
\begin{frame}
\begin{center}
\Huge{{Questions/Answers}}
\end{center}
\end{frame}

\end{document}