\documentclass[12pt,a4paper]{ctexart}
\usepackage{ctex}
\usepackage{CJK}
\usepackage{amsmath}
\usepackage{amsfonts}
\usepackage{amssymb}
\usepackage{graphicx}
\usepackage{picinpar}
\usepackage{textcomp}
\usepackage{faktor}
\usepackage{multirow}
\usepackage{abstract}
\usepackage{hyperref}
\usepackage{graphics}
\usepackage{graphicx}
\usepackage{appendix}
\usepackage{booktabs}
\usepackage{indentfirst}

\addtolength{\topmargin}{-54pt}
\setlength{\oddsidemargin}{0cm} % 3.17cm - 1 inch
\setlength{\evensidemargin}{\oddsidemargin}
\setlength{\textwidth}{17cm}
\setlength{\textheight}{25cm} % 24.62


%\renewcommand\refname{参考文献}
%\renewcommand\figurename{图}
%\renewcommand\tablename{表}

\begin{document}

\title{气体中的声速测量}
\author{陆思锐\thanks{清华大学物理系\quad 基科52班 \quad 2015012206} } 
\date{}
\maketitle
\begin{center}
\begin{tabular}{l r}
实验时间: & October 27, 2015 \\ 
% Date the experiment was performed
%Partners: & James Smith \\ % Partner names
%& Mary Smith \\
指导老师: & 朱美红,郭旭波 % Instructor/supervisor
\end{tabular}
\end{center}

\begin{abstract}
作为机械波,声音的传播实际上是在弹性介质中振动状态的传播,因此声音的速度也取决于传递它的介质。本实验通过行波近似下的相位比较法和驻波假设下的振幅极值法对声速进行测量,并与理论值进行对比,了解声波在空气中传播速度与空气状态参量的关系,掌握在液体和气体中测量声速的装置和基本方法。


\textbf{关键词: 声速, 声波,行波,驻波}
\end{abstract}
\clearpage
\section{实验数据}
无论是行波法还是驻波法,都会得到一组记录卡尺位置的数据$l_j$以及对应的序号$j$.在处理数据时应计算以下回归直线的斜率$b_1$:
$$l_j = b_0 + b_1j$$
线性拟合可以得到不确定度,根据$b_1$就可以得到声波的波长。

频率$f$为读出,不确定度为
$$U_f = 0.05\%f + 2(\delta f )$$
其中$\delta f$是数字显示末位等于“1”时的量值。在本实验中,数字显示的末位为百分之一赫兹,所以$\delta f =0.01Hz$.
最终的声速不确定度就是波长不确定度与频率不确定度的复合
$$U_v/v = \sqrt{(U_{\lambda}/\lambda)^2 +(U_f/f)^2}$$

\subsection{行波近似下的相位比较法}
频率读出为
$$f=40830.0Hz$$
相对不确定度按公式计算为
$$U_f = 0.05\%f + 2(\delta f )=20.615Hz$$
相对不确定度为$0.0504\%$,于是频率$f$为
$$f=(4.083\pm0.002)\times 10^4 Hz$$
% Table generated by Excel2LaTeX from sheet 'Sheet2'
\begin{table}[htbp]
  \centering
  \caption{行波近似下的相位比较法}
\scalebox{0.75}{
    \begin{tabular}{rrrrrrrrrrrrrrrr}
    \toprule
    y/mm & 0.44  & 9.56  & 18.07 & 26.8  & 35.43 & 43.93 & 52.39 & 60.97 & 69.22 & 77.49 & 86.13 & 94.85 & 103.11 & 111.31 & 119.33 \\
    \midrule
    数     & 1     & 2     & 3     & 4     & 5     & 6     & 7     & 8     & 9     & 10    & 11    & 12    & 13    & 14    & 15 \\
    \bottomrule
    \end{tabular}%
}
  \label{tab:addlabel}%
\end{table}%

经过数据处理得到:斜率$b_1=8.486357143\times 10^{-3}m$,斜率标准差$s_{b_1}=2.1172527m\times 10^{-5}$
计算t因子为
$$t(0.05,13)=2.160368656$$
于是A类不确定度为$U_{\lambda}=ts_{b_1}=4.573\times 10^{-5}m$、
从而波长$\lambda$为
$$\lambda=(8.486\pm0.045)\times 10^{-3}m$$

相对不确定度为$U_{\lambda}/\lambda=0.53\%$,
根据公式计算出
$$v=\lambda f=346.48m/s$$
$$U_v=v\sqrt{(U_{\lambda}/\lambda)^2 +(U_f/f)^2}=1.844m/s$$
从而声速测量值为
$$v=(346.5\pm1.8) m/s$$
\subsection{驻波假设下的振幅极值法}
频率读出为
$$f=40830.0Hz$$
相对不确定度按公式计算为
$$U_f = 0.05\%f + 2(\delta f )=20.615Hz$$
相对不确定度为$0.0504\%$,于是频率$f$为
$$f=(4.083\pm0.002)\times 10^4 Hz$$
% Table generated by Excel2LaTeX from sheet 'Sheet2'
\begin{table}[htbp]
  \centering
  \caption{驻波假设下的振幅极值法}
    \begin{tabular}{r|r|r|r|r|r|r|r|r|r|r}
    \toprule
    y(cm) & 62.15 & 66.62 & 70.9  & 75.47 & 79.71 & 83.96 & 88.04 & 92.33 & 96.35 & 100.93 \\
    \midrule
    数     & 1    & 2    & 3    & 4    & 4    & 6    & 7    & 8    & 9    & 10 \\
    \bottomrule
    \end{tabular}%
  \label{tab:addlabel}%
\end{table}%


经过数据处理得到:斜率$b_1=4.280242424\times 10^{-3}m$,斜率标准差$s_{b_1}=0.020189825\times 10^{-3}$
计算t因子为
$$t(0.05,8)=2.306$$
于是A类不确定度为$U_{\lambda}=ts_{b_1}=4.655\times 10^{-5}m$、
从而波长$\lambda$为
$$\lambda=(8.560\pm0.092)\times 10^{-3}m$$

相对不确定度为$U_{\lambda}/\lambda=1.07\%$
根据公式计算出
$$v=\lambda f=349.50m/s$$
$$U_v=v\sqrt{(U_{\lambda}/\lambda)^2 +(U_f/f)^2}=3.91m/s$$
从而声速测量值为
$$v=(349.5\pm3.9) m/s$$

\subsection{经验公式值}
实验室中空气温度:$t =24.0$摄氏度,实验前后相对湿度为$32.0\%,33.0\%$,取平均值得到空气相对湿度为:$r =32.5\%$
则饱和蒸汽压为
$$\lg{p_s} =10.286 - 1780/(237.3+t) =3.4739$$
得到$$p_s=32.2625Pa$$
则理论上的声速为
$$v_t = 331.5\sqrt{(1+t/T_0 ) (1+0.16(rp_s)/p)}=345.759 m/s$$

理论值在两个实验值的置信区间里面。
\section{讨论}
\subsection{空气中理论声速与实际声速之间的误差分析}

\begin{enumerate}
\item 理论公式的适用性。理论计算公式适用于理想气体,空气并不是理想气体。
\item 色散现象会导致速度与频率有关。
\item 在数据测量和读取的时候可能会存在偏差。
\end{enumerate}
\subsection{空气中两种声速测量方法的结果差异分析}
\begin{enumerate}
\item 反射声波与入射声波形成的稳定声场并不均匀;并不是平面波。这也是为什么第二个实验误差更大,当$x_i$较小的时候尤其明显,这也是为什么选取了中间一段的数据
\item 声信号强度随传播距离增大而逐渐衰减,因此驻波振幅变化也会与信号衰减有关。
\item 在数据测量和读取的时候可能会存在偏差。
\end{enumerate}
\bibliographystyle{plain}
\nocite{jcwlsyjy}
\bibliography{ref}


\end{document}