% command

% problem

%\newtheorem{problem}[theorem]{Problem}
    %these set up environments for listing things. The numbering is automatic.

    
\newenvironment{sol}[1][Answer]{\noindent\textbf{#1: }}{ \rule{0.5em}{0.5em}

\mbox{}}
    %this is the environment for writing solutions. Doble spaced, with an end of proof
    %box at the end

% Some tools
\newcommand{\enterProblemHeader}[1]{\nobreak\extramarks{#1}{#1 continued on next page\ldots}\nobreak%
                                    \nobreak\extramarks{#1 (continued)}{#1 continued on next page\ldots}\nobreak}%
\newcommand{\exitProblemHeader}[1]{\nobreak\extramarks{#1 (continued)}{#1 continued on next page\ldots}\nobreak%
                                   \nobreak\extramarks{#1}{}\nobreak}%

\newcommand{\homeworkProblemName}{}%
\newcounter{homeworkProblemCounter}%

\newenvironment{prob}[1][Problem \arabic{homeworkProblemCounter}]%
  {\stepcounter{homeworkProblemCounter}
  {\addcontentsline{toc}{subsection}{Problem \arabic{homeworkProblemCounter}}}%
   \renewcommand{\homeworkProblemName}{#1}%
   {\noindent\it \textbf{\homeworkProblemName}.}
   %\subsection*{\homeworkProblemName}%
   \enterProblemHeader{\homeworkProblemName}}%
  {\exitProblemHeader{\homeworkProblemName}}\vspace{15mm}%
  
  

\newcommand{\homeworkSectionName}{}%
\newlength{\homeworkSectionLabelLength}{}%
\newenvironment{homeworkSection}[1]%
  {% We put this space here to make sure we're not connected to the above.

   \renewcommand{\homeworkSectionName}{#1}%
   \settowidth{\homeworkSectionLabelLength}{\homeworkSectionName}%
   \addtolength{\homeworkSectionLabelLength}{0.25in}%
   \changetext{}{-\homeworkSectionLabelLength}{}{}{}%
   \subsection*{\homeworkSectionName}%
   \enterProblemHeader{\homeworkProblemName\ [\homeworkSectionName]}}%
  {\enterProblemHeader{\homeworkProblemName}%



   % We put the blank space above in order to make sure this margin
   % change doesn't happen too soon.
   \changetext{}{+\homeworkSectionLabelLength}{}{}{}}%

%\newcommand{\sol}{\vspace{3mm}\noindent\ {\it \textbf{Solution.}} }

\newcommand{\Acknowledgement}[1]{\ \\{\bf Acknowledgement:} #1}


% template

\newcommand{\Rea}{{\mathbb R}}
\newcommand{\Int}{{\mathbb Z}}
\newcommand{\Rat}{{\mathbb Q}}
\newcommand{\Cmp}{{\mathbb C}}
\newcommand{\Nat}{{\mathbb N}}

% Overleaf Template

\newcommand{\ra}[1]{\renewcommand{\arraystretch}{#1}}

\newtheorem{thm}{Theorem}[section]
\newtheorem{prop}[thm]{Proposition}
\newtheorem{lem}[thm]{Lemma}
\newtheorem{cor}[thm]{Corollary}
\newtheorem{defn}[thm]{Definition}
\newtheorem{rem}[thm]{Remark}
\newtheorem{ques}[thm]{Question}
\numberwithin{equation}{section}



\newcommand{\bbF}{\mathbb{F}}
\newcommand{\bbX}{\mathbb{X}}
\newcommand{\bI}{\mathbf{I}}
\newcommand{\bX}{\mathbf{X}}
\newcommand{\bY}{\mathbf{Y}}
\newcommand{\bepsilon}{\boldsymbol{\epsilon}}
\newcommand{\balpha}{\boldsymbol{\alpha}}
\newcommand{\bbeta}{\boldsymbol{\beta}}
\newcommand{\0}{\mathbf{0}}

% Werner Krauth
% FIXME
%
\newcommand\FIXME[1]{{\bf FIXME \{}#1{\bf \}}}
%
% definition of pi and complex unity
%
\newcommand{\mpi}{\mathrmgreek{p}}
\newcommand{\mi}{\mathrm{i}}
%
% Definitions for text, equation numbers, figures, etc
%
\newcommand{\eq}[1]{eqn~(\ref{#1})}
\newcommand{\eqtwo}[2]{eqns~(\ref{e:#1}) and~(\ref{e:#2})}
\newcommand{\fig}[1]{Fig.~\ref{f:#1}}
\newcommand{\quot}[1]{``#1''}
\newcommand{\tab}[1]{Table~\ref{t:#1}} 
\newcommand{\chap}[1]{Chapter~\ref{c:#1}} 
\newcommand{\sect}[1]{Section~\ref{s:#1}} 
\newcommand{\SECT}[1]{\ref{s:#1}} 
\newcommand{\eg}{\textrm{e.g.}}
\newcommand{\cf}{\textrm{cf}}
\newcommand{\etc}{\textrm{etc.}}
\newcommand{\etcp}{\textrm{etc}}
\newcommand{\ie}{\textrm{i.e.}}
\newcommand{\vs}{\textrm{vs.}}
%
% text-indices
%
\newcommand{\betac}{{\beta_{\text{c}}}}  %  Critical inverse temperature
\newcommand{\Tc}{{T_{\text{c}}}}  %  Critical temperature
\newcommand{\Tr}{\text{Tr}}  %  Trace
%
%  non-diagonal density matrix , with optional parameter for `sym' `box' `cube' etc
%
\newcommand{\rhomat}[4][]{\rho^{\text{#1}\!}\lc #2,#3,#4\rc}
%
%
\newcommand{\dos}[2][]{\mathcal{N}^{#1} \! \glb #2 \grb} %  density of states
\newcommand{\dosa}[2][]{\mathcal{N}^{#1}_{#2}} %  density of states
\newcommand{\dosb}[2][]{\mathcal{N}^{#1} \! \glb #2 \grb} %  density of states
\newcommand{\dosc}[2][]{\mathcal{N}^{#1} \! \glc #2 \grc} %  density of states
%
\newcommand{\ACAL}{\mathcal{A}}  %  mathcal 
\newcommand{\BCAL}{\mathcal{B}}  %  mathcal
\newcommand{\CCAL}{\mathcal{C}}  %  mathcal
\newcommand{\DCAL}{\mathcal{D}}  %  mathcal
\newcommand{\ECAL}{\mathcal{E}}  %  mathcal
\newcommand{\FCAL}{\mathcal{F}}  %  mathcal
\newcommand{\GCAL}{\mathcal{G}}  %  mathcal
\newcommand{\MCAL}{\mathcal{M}}  %  mathcal
\newcommand{\NCAL}{\mathcal{N}}  %  mathcal
\newcommand{\OCAL}{\mathcal{O}}  %  mathcal
\newcommand{\PCAL}{\mathcal{P}}  %  mathcal
\newcommand{\QCAL}{\mathcal{Q}}  %  mathcal
\newcommand{\RCAL}{\mathcal{R}}  %  mathcal
\newcommand{\SCAL}{\mathcal{S}}  %  mathcal
\newcommand{\TCAL}{\mathcal{T}}  %  mathcal
\newcommand{\VCAL}{\mathcal{V}}  %  mathcal
\newcommand{\XCAL}{\mathcal{X}}  %  mathcal
\newcommand{\ZCAL}{\mathcal{Z}}  %  mathcal
%
% the common Greek letters and their bold versions
%
\newcommand{\s}{\sigma}
\newcommand{\stilde}{\tilde{\sigma}}
\newcommand{\bs}{\boldsymbol{\sigma}}
\newcommand{\eps}{\epsilon}
%
% poor man's bold for + and - (used for spins)
%
\newcommand{\minus}{\ensuremath{\pmb{-}}}
\newcommand{\plus}{\ensuremath{\pmb{+}}}
%
% Definition for hidden text comments 
%
%
% Definition for { text } in equations (in columnar style)
%
\newcommand{\eqntext}[1]{ \left\{\!\!\!\mbox{\begin{tabular}{c}  #1 \end{tabular}}\!\!\! \right\}}
%
% Definitions for display math
%
\newcommand{\BRA}[1]{\ensuremath{\langle #1 |}} % Dirac notation 
\newcommand{\BRAN}[1]{\ensuremath{\langle #1 }} % Dirac notation  Bra without |
\newcommand{\KET}[1]{\ensuremath{|#1 \rangle }} % Dirac notation
%
% exponentials with braces 'a': no brace 'b' () 'c' [] 'd' {}
%
\newcommand{\expa}[1]{\mathrm{e}^{#1}}   % high exponential groupings a
\newcommand{\expb}[1]{\exp \glb #1 \grb} % low exponential with groupings b
\newcommand{\expba}[1]{\exp \bigl( #1 \bigr)} % low exponential with groupings b-small
\newcommand{\expbb}[1]{\exp \Bigl( #1 \Bigr)} % low exponential with groupings b-medium
\newcommand{\expbc}[1]{\exp \biggl( #1 \biggr)} % low exponential with groupings b-large
\newcommand{\expbd}[1]{\exp \Biggl( #1 \Biggr)} % low exponential with groupings b-Xlarge
\newcommand{\expc}[1]{\exp \glc #1 \grc} % low exponential with groupings c
\newcommand{\expd}[1]{\exp \gld #1 \grd} % low exponential with groupings d
%
% trigonometric functions with braces 'a': no brace 'b' () 'c' [] 'd' {}
%
\newcommand{\sina}[2][]{\sin^{#1} \! \gla #2 \gra}  % sin-brace,  with - nothing
\newcommand{\cosa}[2][]{\cos^{#1} \! \gla #2 \gra}  % cos-brace,  with - nothing
\newcommand{\tana}[2][]{\tan^{#1} \!\gla #2 \gra}  % tan-brace,  with - nothing
\newcommand{\cota}[2][]{\cot^{#1} \!\gla #2 \gra}  % cot-brace,  with - nothing
\newcommand{\sinha}[2][]{\sinh^{#1}\! \gla #2 \gra} % sinh-brace, with - nothing
\newcommand{\cosha}[2][]{\cosh^{#1}\! \gla #2 \gra} % cosh-brace, with - nothing
\newcommand{\tanha}[2][]{\tanh^{#1}\! \gla #2 \gra} % tanh-brace, with - nothing
\newcommand{\cotha}[2][]{\coth^{#1}\! \gla #2 \gra} % coth-brace, with - nothing

\newcommand{\sinb}[2][]{\sin^{#1} \glb #2 \grb}  % sin-brace,  with - ()
\newcommand{\cosb}[2][]{\cos^{#1} \glb #2 \grb}  % cos-brace,  with - ()
\newcommand{\tanb}[2][]{\tan^{#1} \glb #2 \grb}  % tan-brace,  with - ()
\newcommand{\cotb}[2][]{\cot^{#1} \glb #2 \grb}  % cot-brace,  with - ()
\newcommand{\sinhb}[2][]{\sinh^{#1} \glb #2 \grb} % sinh-brace, with - ()
\newcommand{\coshb}[2][]{\cosh^{#1} \glb #2 \grb} % cosh-brace, with - ()
\newcommand{\tanhb}[2][]{\tanh^{#1} \glb #2 \grb} % tanh-brace, with - ()
\newcommand{\cothb}[2][]{\coth^{#1} \glb #2 \grb} % coth-brace, with - ()

\newcommand{\sinc}[2][]{\sin^{#1} \glc #2 \grc}  % sin-brace,  with - []
\newcommand{\cosc}[2][]{\cos^{#1} \glc #2 \grc}  % cos-brace,  with - []
\newcommand{\tanc}[2][]{\tan^{#1} \glc #2 \grc}  % tan-brace,  with - []
\newcommand{\cotc}[2][]{\cot^{#1} \glc #2 \grc}  % cot-brace,  with - []
\newcommand{\sinhc}[2][]{\sinh^{#1} \glc #2 \grc} % sinh-brace, with - []
\newcommand{\coshc}[2][]{\cosh^{#1} \glc #2 \grc} % cosh-brace, with - []
\newcommand{\tanhc}[2][]{\tanh^{#1} \glc #2 \grc} % tanh-brace, with - []
\newcommand{\cothc}[2][]{\coth^{#1} \glc #2 \grc} % coth-brace, with - []

\newcommand{\sind}[2][]{\sin^{#1} \gld #2 \grd}  % sin-brace,  with - {}
\newcommand{\cosd}[2][]{\cos^{#1} \gld #2 \grd}  % cos-brace,  with - {}
\newcommand{\tand}[2][]{\tan^{#1} \gld #2 \grd}  % tan-brace,  with - {}
\newcommand{\cotd}[2][]{\cot^{#1} \gld #2 \grd}  % cot-brace,  with - {}
\newcommand{\sinhd}[2][]{\sinh^{#1} \gld #2 \grd} % sinh-brace, with - {}
\newcommand{\coshd}[2][]{\coth^{#1} \gld #2 \grd} % cosh-brace, with - {}
\newcommand{\tanhd}[2][]{\tanh^{#1} \gld #2 \grd} % tanh-brace, with - {}
\newcommand{\cothd}[2][]{\coth^{#1} \gld #2 \grd} % coth-brace, with - {}

\newcommand{\loga}[2][]{\log^{#1}\! \gla #2 \gra}  % log-brace,  with - nothing
\newcommand{\logb}[2][]{\log^{#1} \glb #2 \grb}  % log-brace,  with - ()
\newcommand{\logc}[2][]{\log^{#1} \glc #2 \grc}  % log-brace,  with - []
\newcommand{\logd}[2][]{\log^{#1} \gld #2 \grd}  % log-brace,  with - {}

\newcommand{\arccosa}[2][]{\arccos^{#1}\! \gla #2 \gra}  % arccos-brace,  with - nothing
\newcommand{\arccosb}[2][]{\arccos^{#1} \glb #2 \grb}  % arccos-brace,  with - ()
\newcommand{\arccosc}[2][]{\arccos^{#1} \glc #2 \grc}  % arccos-brace,  with - []
\newcommand{\arccosd}[2][]{\arccos^{#1} \gld #2 \grd}  % arccos-brace,  with - {}


\newcommand{\arctana}[2][]{\arctan^{#1}\! \gla #2 \gra}  % arctan-brace,  with - nothing
\newcommand{\arctanb}[2][]{\arctan^{#1} \glb #2 \grb}  % arctan-brace,  with - ()
\newcommand{\arctanc}[2][]{\arctan^{#1} \glc #2 \grc}  % arctan-brace,  with - []
\newcommand{\arctand}[2][]{\arctan^{#1} \gld #2 \grd}  % arctan-brace,  with - {}

\newcommand{\gl}{\left(}  % ' group left' 
\newcommand{\gr}{\right)}  % ' group right' 
\newcommand{\lb}{\left[}  % left brace `[' used in sinh[3 beta] etc
\newcommand{\rb}{\right]}  % right brace ']' "           " 
\newcommand{\lc}{\left(}  % left brace `(' used in groupings of expressions
\newcommand{\rc}{\right)}  % right brace ')' "         "            " 
\newcommand{\mult}{\times} % multiplication sign in display math, 
\newcommand{\multcc}{\cdot} % multiplication: character times character
\newcommand{\multnn}{\times} % multiplication:    number times number 
\newcommand{\multcn}{\cdot} % multiplication: character times number
\newcommand{\ran}[2]{\FUNCTION[#1,#2]{ran}}
\newcommand{\nran}[2]{\FUNCTION[#1,#2]{nran}}
\newcommand{\perL}{\text{per,$L$}}
\newcommand{\perLL}{\text{per,$2L$}}
\newcommand{\boxx}[1]{\text{box,$#1$}}

%
% Groupings on levels a--nothing, b--(), c--[], and d--{} (better solution)
%
\newcommand{\gla}{\,}  % ' group left a' 
\newcommand{\gra}{}  % ' group right a' 
\newcommand{\glb}{\left(}  % ' group left b' 
\newcommand{\grb}{\right)}  % ' group right b' 
\newcommand{\glc}{\left[}  % ' group left c'
\newcommand{\grc}{\right]}  % ' group right c' 
\newcommand{\gld}{\left\{}  % ' group left d' 
\newcommand{\grd}{\right\}}  % ' group right d' 

\newcommand{\const}{\text{const}}
\newcommand{\Nhits}{N_{\text{hits}}}
\newcommand{\gauss}[1]{\FUNCTION[#1]{gauss}}
\newcommand{\PLUSPLUS}{+ \dots +}
\newcommand{\PLUSDOTS}{+ \cdots}
\newcommand{\MINUSPLUS}{- \dots +}
\newcommand{\MINUSDOTS}{- \cdots}
\newcommand{\PLUSMINUS}{+ \dots -}
\newcommand{\MINUSMINUS}{- \dots -}
\newcommand{\MULTMULT}{\cdot \cdots \cdot}
\newcommand{\TIMESTIMES}{\times \cdots \times}
\newcommand{\TO}{,\ldots,}
\newcommand{\VEC}[1]{\mathbf{#1}}
%
% vector notations
%
\newcommand{\avec}{\VEC{a}}
\newcommand{\Avec}{\VEC{A}}
\newcommand{\bvec}{\VEC{b}}
\newcommand{\cvec}{\VEC{c}}
\newcommand{\Cvec}{\VEC{C}}
\newcommand{\dvec}{\VEC{d}}
\newcommand{\evec}{\VEC{e}}
\newcommand{\ehatvec}{\hat{\VEC{e}}}
\newcommand{\fvec}{\VEC{f}}
\newcommand{\Fvec}{\VEC{F}}
\newcommand{\gvec}{\VEC{g}}
\newcommand{\hvec}{\VEC{h}}
\newcommand{\ivec}{\VEC{i}}
\newcommand{\jvec}{\VEC{j}}
\newcommand{\kvec}{\VEC{k}}
\newcommand{\lvec}{\VEC{l}}
\newcommand{\mvec}{\VEC{m}}
\newcommand{\nvec}{\VEC{n}}
\newcommand{\ovec}{\VEC{o}}
\newcommand{\pvec}{\VEC{p}}
\newcommand{\qvec}{\VEC{q}}
\newcommand{\rvec}{\VEC{r}}
\newcommand{\Rvec}{\VEC{R}}
\newcommand{\svec}{\VEC{s}}
\newcommand{\tvec}{\VEC{t}}
\newcommand{\uvec}{\VEC{u}}
\newcommand{\vvec}{\VEC{v}}
\newcommand{\wvec}{\VEC{w}}
\newcommand{\xvec}{\VEC{x}}
\newcommand{\Xvec}{\VEC{X}}
\newcommand{\yvec}{\VEC{y}}
\newcommand{\Yvec}{\VEC{Y}}
\newcommand{\zvec}{\VEC{z}}
\newcommand{\Zvec}{\VEC{Z}}
\newcommand{\atilde}{\tilde{a}}
\newcommand{\btilde}{\tilde{b}}
\newcommand{\ctilde}{\tilde{c}}
\newcommand{\ptilde}{\tilde{p}}
\newcommand{\vtilde}{\tilde{v}}
\newcommand{\xtilde}{\tilde{x}}
\newcommand{\ytilde}{\tilde{y}}
\newcommand{\ztilde}{\tilde{z}}
\newcommand{\Etilde}{\tilde{E}}
\newcommand{\Rtilde}{\tilde{R}}
\newcommand{\phitilde}{\tilde{\phi}}
\newcommand{\Psitilde}{\tilde{\Psi}}
\newcommand{\psitilde}{\tilde{\psi}}
\newcommand{\sigmatilde}{\tilde{\sigma}}
\newcommand{\alphavec}{\boldsymbol{\alpha}}
\newcommand{\betavec}{\boldsymbol{\beta}}
\newcommand{\gammavec}{\boldsymbol{\gamma}}
\newcommand{\deltavec}{\boldsymbol{\delta}}
\newcommand{\pivec}{\boldsymbol{\pi}}
\newcommand{\epsvec}{\boldsymbol{\epsilon}}
\newcommand{\sigmavec}{\boldsymbol{\sigma}}
\newcommand{\del}{\delta}
\newcommand{\dd}[1]{\text{d}{#1\ }}   % this is for differentials in formulas
\newcommand{\ddd}[1]{\text{d}{#1}}   % same as above, without space 
%
% Delta_tau Delta_x Delta_y, etc
%
\newcommand{\Delbeta}{\Delta_{\beta}}
\newcommand{\DelE}{\Delta_E}
\newcommand{\Deli}{\Delta_i}
\newcommand{\Delk}{\Delta_k}
\newcommand{\Dellambda}{\Delta_{\lambda}}
\newcommand{\Delmu}{\Delta_{\mu}}
\newcommand{\DelM}{\Delta_{M}}
\newcommand{\Deln}{\Delta_n}
\newcommand{\Delphi}{\Delta_{\phi}}
\newcommand{\Delr}{\Delta_r}
\newcommand{\Delt}{\Delta_t}
\newcommand{\Deltau}{\Delta_{\tau}}
\newcommand{\Delvvec}{\Delta_{\vvec}}
\newcommand{\Delx}{\Delta_x}
\newcommand{\Delxp}{\Delta'_x}
\newcommand{\Delxk}{\Delta_{x_k}}
\newcommand{\Delxvec}{\Delta_{\xvec}}
\newcommand{\Delxvecp}{\Delta_{\xvec'}}
\newcommand{\Dely}{\Delta_y}
\newcommand{\Delyvec}{\Delta_{\yvec}}
\newcommand{\Delz}{\Delta_z}
%
% scalar product
%
\newcommand{\neigh}[2]{\langle #1 , #2 \rangle}
\newcommand{\scal}[2]{(#1 \pmb{\cdot} #2)}
\newcommand{\mean}[1]{\left\langle #1 \right\rangle}
\newcommand{\half}{\frac{1}{2}}
\newcommand{\thalf}{\tfrac{1}{2}}

\newcommand{\SET}[1]{\{#1\}}
