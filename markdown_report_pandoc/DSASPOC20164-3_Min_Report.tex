\documentclass[12pt,a4paper]{ctexart}
\usepackage{geometry}     % 設定邊界
\geometry{
  top=1in,
  inner=1in,
  outer=1in,
  bottom=1in,
  headheight=3ex,
  headsep=2ex
}
\usepackage[T1]{fontenc}
\usepackage{lmodern}
\usepackage{xeCJK}
\usepackage{amssymb,amsmath}
\usepackage{ifxetex,ifluatex}
\usepackage{indentfirst}
\def\tightlist{}
\usepackage{fixltx2e} % provides \textsubscript
% use upquote if available, for straight quotes in verbatim environments
\IfFileExists{upquote.sty}{\usepackage{upquote}}{}

% use microtype if available
\IfFileExists{microtype.sty}{\usepackage{microtype}}{}
\usepackage{color}
\usepackage{fancyvrb}
\newcommand{\VerbBar}{|}
\newcommand{\VERB}{\Verb[commandchars=\\\{\}]}
\DefineVerbatimEnvironment{Highlighting}{Verbatim}{commandchars=\\\{\}}
% Add ',fontsize=\small' for more characters per line
\newenvironment{Shaded}{}{}
\newcommand{\KeywordTok}[1]{\textcolor[rgb]{0.00,0.44,0.13}{\textbf{{#1}}}}
\newcommand{\DataTypeTok}[1]{\textcolor[rgb]{0.56,0.13,0.00}{{#1}}}
\newcommand{\DecValTok}[1]{\textcolor[rgb]{0.25,0.63,0.44}{{#1}}}
\newcommand{\BaseNTok}[1]{\textcolor[rgb]{0.25,0.63,0.44}{{#1}}}
\newcommand{\FloatTok}[1]{\textcolor[rgb]{0.25,0.63,0.44}{{#1}}}
\newcommand{\ConstantTok}[1]{\textcolor[rgb]{0.53,0.00,0.00}{{#1}}}
\newcommand{\CharTok}[1]{\textcolor[rgb]{0.25,0.44,0.63}{{#1}}}
\newcommand{\SpecialCharTok}[1]{\textcolor[rgb]{0.25,0.44,0.63}{{#1}}}
\newcommand{\StringTok}[1]{\textcolor[rgb]{0.25,0.44,0.63}{{#1}}}
\newcommand{\VerbatimStringTok}[1]{\textcolor[rgb]{0.25,0.44,0.63}{{#1}}}
\newcommand{\SpecialStringTok}[1]{\textcolor[rgb]{0.73,0.40,0.53}{{#1}}}
\newcommand{\ImportTok}[1]{{#1}}
\newcommand{\CommentTok}[1]{\textcolor[rgb]{0.38,0.63,0.69}{\textit{{#1}}}}
\newcommand{\DocumentationTok}[1]{\textcolor[rgb]{0.73,0.13,0.13}{\textit{{#1}}}}
\newcommand{\AnnotationTok}[1]{\textcolor[rgb]{0.38,0.63,0.69}{\textbf{\textit{{#1}}}}}
\newcommand{\CommentVarTok}[1]{\textcolor[rgb]{0.38,0.63,0.69}{\textbf{\textit{{#1}}}}}
\newcommand{\OtherTok}[1]{\textcolor[rgb]{0.00,0.44,0.13}{{#1}}}
\newcommand{\FunctionTok}[1]{\textcolor[rgb]{0.02,0.16,0.49}{{#1}}}
\newcommand{\VariableTok}[1]{\textcolor[rgb]{0.10,0.09,0.49}{{#1}}}
\newcommand{\ControlFlowTok}[1]{\textcolor[rgb]{0.00,0.44,0.13}{\textbf{{#1}}}}
\newcommand{\OperatorTok}[1]{\textcolor[rgb]{0.40,0.40,0.40}{{#1}}}
\newcommand{\BuiltInTok}[1]{{#1}}
\newcommand{\ExtensionTok}[1]{{#1}}
\newcommand{\PreprocessorTok}[1]{\textcolor[rgb]{0.74,0.48,0.00}{{#1}}}
\newcommand{\AttributeTok}[1]{\textcolor[rgb]{0.49,0.56,0.16}{{#1}}}
\newcommand{\RegionMarkerTok}[1]{{#1}}
\newcommand{\InformationTok}[1]{\textcolor[rgb]{0.38,0.63,0.69}{\textbf{\textit{{#1}}}}}
\newcommand{\WarningTok}[1]{\textcolor[rgb]{0.38,0.63,0.69}{\textbf{\textit{{#1}}}}}
\newcommand{\AlertTok}[1]{\textcolor[rgb]{1.00,0.00,0.00}{\textbf{{#1}}}}
\newcommand{\ErrorTok}[1]{\textcolor[rgb]{1.00,0.00,0.00}{\textbf{{#1}}}}
\newcommand{\NormalTok}[1]{{#1}}
\ifxetex
  \usepackage[setpagesize=false, % page size defined by xetex
              unicode=false, % unicode breaks when used with xetex
              xetex]{hyperref}
\else
  \usepackage[unicode=true]{hyperref}
\fi
\hypersetup{breaklinks=true,
            bookmarks=true,
            pdfauthor={陆思锐},
            pdftitle={DSASPOC2016 4-3 Min 解题报告},
            colorlinks=true,
            urlcolor=blue,
            linkcolor=magenta,
            pdfborder={0 0 0}}
\urlstyle{same}  % don't use monospace font for urls
%\setlength{\parindent}{0pt}
%\setlength{\parskip}{6pt plus 2pt minus 1pt}
\setlength{\emergencystretch}{3em}  % prevent overfull lines

%\usepackage{titling}
%\setlength{\droptitle}{-8em}  % 將標題移動至頁面的上面

\usepackage{fancyhdr}
\usepackage{lastpage}
\pagestyle{fancyplain}

\setcounter{secnumdepth}{5}

\title{DSASPOC2016 4-3 Min 解题报告}
\author{陆思锐\thanks{清华大学物理系\quad 基科52班 \quad 2015012206} }




\begin{document}

\maketitle



\begin{abstract}
本文系2016年春季学期邓俊辉老师的 SPOC 数据结构课第四次 PA 第三题 Min
的解题报告. 给出了一种用堆的基本应用(下滤、堆排序)的解法。

\textbf{关键词:} 数据结构, 算法, 解题报告, 完全二叉堆, 堆排序
\end{abstract}



\tableofcontents
\section{题目}\label{ux9898ux76ee}

\subsection{描述}\label{ux63cfux8ff0}

目前我们已经进入了大数据(Big Data)时代,所

\subsection{输入}\label{ux8f93ux5165}

为了加快输入输出速度,输入文件采用二进制格式存储。前两个数字n和k,表示你需要从n个数字中选出最小的k个。之后随即连续存储了n个整数。

\subsection{输出}\label{ux8f93ux51fa}

输出包括k行,依次为\textbf{从大到小}的最小的k个数字,考虑到输入中的数字可能相同,这里输出的数字也可能相同。

\subsection{输入样例1}\label{ux8f93ux5165ux6837ux4f8b1}

\begin{verbatim}
6
3
1
4
-5
1
4
-2
\end{verbatim}

\subsection{输出样例1}\label{ux8f93ux51faux6837ux4f8b1}

\begin{verbatim}
1
-2
-5
\end{verbatim}

\subsection{限制}\label{ux9650ux5236}

1 \textless{}= n \textless{}= 10,000,000

\subsection{提示}\label{ux63d0ux793a}

自己测试时,如果对读入二进制数据不熟悉,可以以文本的形式读取数据;

\section{解题报告}\label{ux89e3ux9898ux62a5ux544a}

\subsection{数据范围}\label{ux6570ux636eux8303ux56f4}

根据提示

\subsection{样例分析}\label{ux6837ux4f8bux5206ux6790}

字面意思.

\subsection{算法设计}\label{ux7b97ux6cd5ux8bbeux8ba1}

有什么数据结构可以支持对快速的极值维护呢?

\subsection{数据结构设计}\label{ux6570ux636eux7ed3ux6784ux8bbeux8ba1}

\begin{enumerate}
\def\labelenumi{\arabic{enumi}.}
\tightlist
\item
  参考邓老师的模板实现了一个堆(支持下滤堆排序等等)
\end{enumerate}

\subsection{问题}\label{ux95eeux9898}

一开始使用堆的类型使用的是泛型,

\subsection{时间和空间复杂度的估算}\label{ux65f6ux95f4ux548cux7a7aux95f4ux590dux6742ux5ea6ux7684ux4f30ux7b97}

\subsection{吻合之处}\label{ux543bux5408ux4e4bux5904}

输入量非常大, 根据提示采用了最快的 fread 方式输入输出。用时最高784ms

\section{代码}\label{ux4ee3ux7801}

\begin{Shaded}
\begin{Highlighting}[]
\OtherTok{#include<cstdio>}
\OtherTok{#include<iostream>}
\end{Highlighting}
\end{Shaded}

\section{参考文献}\label{ux53c2ux8003ux6587ux732e}

\begin{enumerate}
\def\labelenumi{\arabic{enumi}.}
\tightlist
\item
  邓俊辉老师数据结构课件
\item
  提示1
\item
  邓俊辉老师heap 相关代码
\end{enumerate}

\end{document}