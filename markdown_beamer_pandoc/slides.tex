\documentclass[ignorenonframetext,]{beamer}


\setbeamertemplate{caption}[numbered]
\setbeamertemplate{caption label separator}{: }
\setbeamercolor{caption name}{fg=normal text.fg}
\beamertemplatenavigationsymbolsempty
\usepackage{lmodern}

\usepackage{beamerthemeTsinghua}

%\mode<presentation> {
%\usetheme{Rochester}
%\usecolortheme{seahorse}
%}

\usepackage{xeCJK}
\usepackage{amssymb,amsmath}
\usepackage{ifxetex,ifluatex}
\usepackage{fixltx2e} % provides \textsubscript


\newif\ifbibliography
\usepackage{natbib}
\bibliographystyle{plainnat}
\hypersetup{
            pdftitle={Beamer template},
            pdfauthor={Sirui Lu},
            pdfborder={0 0 0},
            breaklinks=true}
\urlstyle{same}  % don't use monospace font for urls
\usepackage{longtable,booktabs}
\usepackage{caption}
% These lines are needed to make table captions work with longtable:
\makeatletter
\def\fnum@table{\tablename~\thetable}
\makeatother





% Prevent slide breaks in the middle of a paragraph:
\widowpenalties 1 10000
\raggedbottom

\AtBeginPart{
  \let\insertpartnumber\relax
  \let\partname\relax
  \frame{\partpage}
}
\AtBeginSection{
  \ifbibliography
  \else
    \let\insertsectionnumber\relax
    \let\sectionname\relax
    \frame{\sectionpage}
  \fi
}
\AtBeginSubsection{
  \let\insertsubsectionnumber\relax
  \let\subsectionname\relax
  \frame{\subsectionpage}
}

\setlength{\parindent}{0pt}
\setlength{\parskip}{6pt plus 2pt minus 1pt}
\setlength{\emergencystretch}{3em}  % prevent overfull lines
\providecommand{\tightlist}{%
  \setlength{\itemsep}{0pt}\setlength{\parskip}{0pt}}
\setcounter{secnumdepth}{0}

\title{Beamer template}
\author{Sirui Lu}
\institute{Department of Physics, Tsinghua University}
\date{March 5, 2017}
\logo{\includegraphics[width=1cm]{logo.eps}}

\begin{document}
\frame{\titlepage}

\begin{frame}
\frametitle{Outline} % Table of contents slide, comment this block out to remove it
\tableofcontents% Throughout your presentation, if you choose to use \section{} and \subsection{} commands, these will automatically be printed on this slide as an overview of your presentation
\end{frame}

\section{量子物理简介}\label{ux91cfux5b50ux7269ux7406ux7b80ux4ecb}

\begin{frame}{维度诅咒}

\begin{enumerate}
\def\labelenumi{\arabic{enumi}.}
\tightlist
\item
  单粒子: \(c_0|0\rangle+c_1|1\rangle\in H\) Dimension is 2
\item
  双粒子:
\end{enumerate}

\begin{itemize}
\tightlist
\item
  \(c_{00}|00\rangle+c_{01}|01\rangle+c_{10}|10\rangle+c_{11}|11\rangle\in H \otimes H\)
  Dimension is 4
\item
  \(|00\rangle+|11\rangle\): \textbf{纠缠!}
\end{itemize}

\begin{enumerate}
\def\labelenumi{\arabic{enumi}.}
\setcounter{enumi}{2}
\tightlist
\item
  多粒子:
\end{enumerate}

\begin{itemize}
\tightlist
\item
  \(c_{0\dots0}|0\cdots0\rangle+\cdots+c_{1\cdots1}|1\cdots1\rangle \in H^{\otimes n}\)
\item
  维度数为 \textbf{\(2^n\)} !!!
\end{itemize}

希尔伯特空间太大了!如何有效地描述?

\end{frame}

\section{项目信息}\label{ux9879ux76eeux4fe1ux606f}

\begin{frame}{总体目标}

\begin{itemize}
\tightlist
\item
  吃喝玩乐
\end{itemize}

\end{frame}

\begin{frame}{具体问题样例:}

\begin{itemize}
\tightlist
\item
  吃喝玩乐
\end{itemize}

\end{frame}

\begin{frame}{项目时间节点}

\begin{itemize}
\tightlist
\item
  2017年3月 立项
\item
  2017年4月 完成前期准备工作
\item
  2017年7月 完成现有工作重现
\item
  2018年2月 完成新问题的初步探索
\item
  2018年3月 结题
\end{itemize}

\end{frame}

\begin{frame}{项目需要}

\begin{itemize}
\tightlist
\item
  吃喝玩乐
\end{itemize}

\end{frame}

\section{预算}\label{ux9884ux7b97}

\begin{frame}{调研:}

仅提供G4和G2两种型号:

\begin{itemize}
\tightlist
\item
  吃喝玩乐
\end{itemize}

\end{frame}

\begin{frame}{方案:}

\begin{longtable}[]{@{}lllll@{}}
\toprule
类别 & 硬件(参考) & 单价 & 数量 & 总计\tabularnewline
\midrule
\endhead
可拓展主板 & 技嘉 X99P-SLI 主板 & 3000 & 1 & 3000\tabularnewline
& & & &\tabularnewline
& & & &\tabularnewline
& & & &\tabularnewline
& & & &\tabularnewline
& & & &\tabularnewline
& & & &\tabularnewline
& & & &\tabularnewline
& & & &\tabularnewline
& & & &\tabularnewline
\bottomrule
\end{longtable}

\end{frame}

\section{附录}\label{ux9644ux5f55}

\begin{frame}{项目需要}

本项目唯一的实验条件就是计算资源。

\end{frame}

\renewcommand\refname{Reference}
\begin{frame}[allowframebreaks]{Reference}
\bibliographytrue
\bibliography{ref.bib}
\end{frame}


\begin{frame}
\begin{center}
\Huge{{Thank you!}}
\end{center}
\end{frame}
\begin{frame}
\begin{center}
\Huge{{Questions/Answers}}
\end{center}
\end{frame}

\end{document}