% part only for tex. There are 9 parts of preliminary abbreviations.

%1=general environment

\usepackage{amsmath,amsfonts,amssymb,color,epsfig,graphics,graphicx,latexsym,theorem,url,verbatim}

%\usepackage{fullpage}
\usepackage[cm]{fullpage}

%\usepackage{pdfpages}

%\usepackage{hyperref}

%\usepackage[colorlinks,linkcolor=blue,anchorcolor=red,citecolor=green]{hyperref}

\newtheorem{definition}{Definition}
\newtheorem{proposition}[definition]{Proposition}
\newtheorem{lemma}[definition]{Lemma}
\newtheorem{algorithm}[definition]{Algorithm}
\newtheorem{fact}[definition]{Fact}
\newtheorem{theorem}[definition]{Theorem}
\newtheorem{corollary}[definition]{Corollary}
\newtheorem{conjecture}[definition]{Conjecture}
\newtheorem{postulate}[definition]{Postulate}
\newtheorem{axiom}[definition]{Axiom}
\newtheorem{remark}[definition]{Remark}
\newtheorem{example}[definition]{Example}
\newtheorem{question}[definition]{Question}
\newtheorem{observation}[definition]{Observation}

\def\squareforqed{\hbox{\rlap{$\sqcap$}$\sqcup$}}
\def\qed{\ifmmode\squareforqed\else{\unskip\nobreak\hfil
\penalty50\hskip1em\null\nobreak\hfil\squareforqed
\parfillskip=0pt\finalhyphendemerits=0\endgraf}\fi}
\def\endenv{\ifmmode\;\else{\unskip\nobreak\hfil
\penalty50\hskip1em\null\nobreak\hfil\;
\parfillskip=0pt\finalhyphendemerits=0\endgraf}\fi}
% unavailable for beamer:
\newenvironment{proof}{\noindent \textbf{{Proof.~} }}{\qed}
\def\Dbar{\leavevmode\lower.6ex\hbox to 0pt
{\hskip-.23ex\accent"16\hss}D}
% Define a new 'leo' style for the package that will use a smaller font.
\makeatletter
\def\url@leostyle{%
  \@ifundefined{selectfont}{\def\UrlFont{\sf}}{\def\UrlFont{\small\ttfamily}}}
\makeatother
% Now actually use the newly defined style.
\urlstyle{leo}



\def\bcj{\begin{conjecture}}
\def\ecj{\end{conjecture}}
\def\bcr{\begin{corollary}}
\def\ecr{\end{corollary}}
\def\bd{\begin{definition}}
\def\ed{\end{definition}}
\def\bea{\begin{eqnarray}}
\def\eea{\end{eqnarray}}
\def\bem{\begin{enumerate}}
\def\eem{\end{enumerate}}
\def\bex{\begin{example}}
\def\eex{\end{example}}
\def\bim{\begin{itemize}}
\def\eim{\end{itemize}}
\def\bl{\begin{lemma}}
\def\el{\end{lemma}}
\def\bpf{\begin{proof}}
\def\epf{\end{proof}}
\def\bpp{\begin{proposition}}
\def\epp{\end{proposition}}
\def\bqu{\begin{question}}
\def\equ{\end{question}}
\def\br{\begin{remark}}
\def\er{\end{remark}}
\def\bt{\begin{theorem}}
\def\et{\end{theorem}}

\def\btb{\begin{tabular}}
\def\etb{\end{tabular}}


\newcommand{\nc}{\newcommand}

%2=alphabet

\def\a{\alpha}
\def\b{\beta}
\def\g{\gamma}
\def\d{\delta}
\def\e{\epsilon}
\def\ve{\varepsilon}
\def\z{\zeta}
\def\h{\eta}
\def\t{\theta}
\def\i{\iota}
\def\k{\kappa}
\def\l{\lambda}
\def\m{\mu}
\def\n{\nu}
\def\x{\xi}
\def\p{\pi}
\def\r{\rho}
\def\s{\sigma}
\def\ta{\tau}
\def\u{\upsilon}
\def\ph{\varphi}
\def\c{\chi}
\def\ps{\psi}
\def\o{\omega}

\def\G{\Gamma}
\def\D{\Delta}
\def\T{\Theta}
\def\L{\Lambda}
\def\X{\Xi}
\def\P{\Pi}
\def\S{\Sigma}
\def\U{\Upsilon}
\def\Ph{\Phi}
\def\Ps{\Psi}
\def\O{\Omega}



 \nc{\bA}{{\bf A}} \nc{\bB}{{\bf B}}
 %\nc{\bC}{{\bf C}}
 \nc{\bC}{{\mathbb{C}}}
 \nc{\bD}{{\bf D}} \nc{\bE}{{\bf E}} \nc{\bF}{{\bf F}}
 \nc{\bG}{{\bf G}} \nc{\bH}{{\bf H}} \nc{\bI}{{\bf I}}
 \nc{\bJ}{{\bf J}} \nc{\bK}{{\bf K}} \nc{\bL}{{\bf L}}
 \nc{\bM}{{\bf M}} \nc{\bN}{{\bf N}} \nc{\bO}{{\bf O}}
 \nc{\bP}{{\bf P}} \nc{\bQ}{{\bf Q}} \nc{\bR}{{\bf R}}
 \nc{\bS}{{\bf S}} \nc{\bT}{{\bf T}} \nc{\bU}{{\bf U}}
 \nc{\bV}{{\bf V}} \nc{\bW}{{\bf W}} \nc{\bX}{{\bf X}}
 \nc{\bZ}{{\bf Z}}

%\bQ, \bR, \bZ denotes the set of rational, real and integer numbers.

\nc{\cA}{{\cal A}} \nc{\cB}{{\cal B}} \nc{\cC}{{\cal C}}
\nc{\cD}{{\cal D}} \nc{\cE}{{\cal E}} \nc{\cF}{{\cal F}}
\nc{\cG}{{\cal G}} \nc{\cH}{{\cal H}} \nc{\cI}{{\cal I}}
\nc{\cJ}{{\cal J}} \nc{\cK}{{\cal K}} \nc{\cL}{{\cal L}}
\nc{\cM}{{\cal M}} \nc{\cN}{{\cal N}} \nc{\cO}{{\cal O}}
\nc{\cP}{{\cal P}} \nc{\cQ}{{\cal Q}} \nc{\cR}{{\cal R}}
\nc{\cS}{{\cal S}} \nc{\cT}{{\cal T}} \nc{\cU}{{\cal U}}
\nc{\cV}{{\cal V}} \nc{\cW}{{\cal W}} \nc{\cX}{{\cal X}}
\nc{\cZ}{{\cal Z}}

% \cX denotes a set, etc in mathematical definition.

\nc{\hA}{{\hat{A}}} \nc{\hB}{{\hat{B}}} \nc{\hC}{{\hat{C}}}
\nc{\hD}{{\hat{D}}} \nc{\hE}{{\hat{E}}} \nc{\hF}{{\hat{F}}}
\nc{\hG}{{\hat{G}}} \nc{\hH}{{\hat{H}}} \nc{\hI}{{\hat{I}}}
\nc{\hJ}{{\hat{J}}} \nc{\hK}{{\hat{K}}} \nc{\hL}{{\hat{L}}}
\nc{\hM}{{\hat{M}}} \nc{\hN}{{\hat{N}}} \nc{\hO}{{\hat{O}}}
\nc{\hP}{{\hat{P}}} \nc{\hR}{{\hat{R}}} \nc{\hS}{{\hat{S}}}
\nc{\hT}{{\hat{T}}} \nc{\hU}{{\hat{U}}} \nc{\hV}{{\hat{V}}}
\nc{\hW}{{\hat{W}}} \nc{\hX}{{\hat{X}}} \nc{\hZ}{{\hat{Z}}}

\nc{\hn}{{\hat{n}}}

%3=math symbol, personal

%3.1 tensor rank

\def\cd{\mathop{\rm CD}}
% canonical decomposition, namely the convex sum of r product states
\def\cdr{\mathop{\rm CD_R}}
% canonical decomposition over the real field
\def\scd{\mathop{\rm SCD}}
% symmetric canonical decomposition, namely the convex sum of r symmetric product states
\def\scdr{\mathop{\rm SCD_R}}
% symmetric canonical decomposition over the real field
\def\ocd{\mathop{\rm OCD}}
% orthogonal canonical decomposition, namely the convex sum of r orthogonal product states
\def\socd{\mathop{\rm SOCD}}
% strong orthogonal canonical decomposition, namely the convex sum of r locally orthogonal product states

\def\rk{\mathop{\rm rk}}
%rk=tensor rank with canonical decomposition
\def\rkr{\mathop{\rm rk_R}}
%rk=tensor rank with real canonical decomposition
\def\srk{\mathop{\rm srk}}
%srk=symmetric tensor rank with symmetric canonical decomposition
\def\srkr{\mathop{\rm srk_R}}
%srk=symmetric tensor rank with real canonical decomposition

\def\rrk{\mathop{\rm rrk}}
%rrk=regularized tensor rank
\def\rsrk{\mathop{\rm rsrk}}
%rsrk=regularized symmetric tensor rank

\def\grk{\mathop{\rm grk}}
%grk=generic tensor rank equal to the tensor rank of most tensors in the space; there is only one grk
\def\trk{\mathop{\rm trk}}
%trk=typical tensor rank equal to the tensor rank of a part of tensors in the space; there may exist a few different trk
\def\ark{\mathop{\rm ark}}
%ark=asymmetric tensor rank, where decomposition contains at least one asymmetric product states
\def\brk{\mathop{\rm brk}}
%brk=border tensor rank
\def\sbrk{\mathop{\rm sbrk}}
%bsrk=symmetric border tensor rank

\def\ork{\mathop{\rm ork}}
%ork=orthogonal tensor rank
\def\sork{\mathop{\rm sork}}
%sork=strong orthogonal tensor rank



%3.2 general

\def\birank{\mathop{\rm birank}}
%birank=(rank,rank^\G)
\def\cmi{\mathop{\rm CMI}}
\def\co{\mathop{\rm Co}}
\def\cps{\mathop{\rm CPS}}
%cps=closest product state in the geometric measure of entanglement
\def\css{\mathop{\rm CSS}}
%cps=closest separable state in the geometric measure of entanglement
\def\csd{\mathop{\rm CSD}}
%csd=canonical separable decomposition, i.e., reaching the length
\def\diag{\mathop{\rm diag}}
\def\dim{\mathop{\rm Dim}}
\def\epr{\mathop{\rm EPR}}
\def\ev{\mathop{\rm EV}}
%EV=eigenvalue
\def\ghz{\mathop{\rm GHZ}}
\def\ghzg{\mathop{\rm GHZg}}
\def\gh{\mathop{\rm GH}}
\def\gz{\mathop{\rm GZ}}
\def\hz{\mathop{\rm HZ}}
\def\I{\mathop{\rm I}}
\def\iso{\mathop{\rm iso}}
\def\lin{\mathop{\rm span}}
\def\loc{\mathop{\rm Loc}}
%Loc=local CPTP map
\def\locc{\mathop{\rm LOCC}}
\def\lu{\mathop{\rm LU}}
\def\max{\mathop{\rm max}}
\def\min{\mathop{\rm min}}
\def\mspec{\mathop{\rm mspec}}
\def\oghz{\mathop{\overline{\ghz}}}
\def\per{\mathop{\rm per}}
\def\ppt{\mathop{\rm PPT}}
\def\pr{\mathop{\rm pr}}
%pr=polynomial rank in algebraic geometry for symmetric states
\def\pro{\mathop{\rm PRO}}
%pro=product states
\def\rank{\mathop{\rm rank}}
\def\sd{\mathop{\rm SD}}
%sd=separable decomposition
\def\sep{\mathop{\rm SEP}}
\def\slocc{\mathop{\rm SLOCC}}
\def\sr{\mathop{\rm sr}}
%sr=Schmidt rank
\def\supp{\mathop{\rm supp}}
\def\tr{\mathop{\rm Tr}}
\def\w{\mathop{\rm W}}
\def\werner{\mathop{\rm werner}}


\def\sp{\mathop{\rm sp}}



\def\GL{{\mbox{\rm GL}}}
\def\PGL{{\mbox{\rm PGL}}}
\def\SL{{\mbox{\rm SL}}}
\def\SO{{\mbox{\rm SO}}}
\def\Ort{{\mbox{\rm O}}}
\def\Un{{\mbox{\rm U}}}

%3.3 abbreviation

\newcommand{\pv}{projective variety}
\newcommand{\pvs}{projective varieties}


%4=math symbol, default

\def\xr{X_\r}
\def\xrg{X_{\r^\G}}
\def\axr{\abs{X_\r}}
\def\axrg{\abs{X_{\r^\G}}}

\def\bigox{\bigotimes}
\def\dg{\dagger}
\def\es{\emptyset}
\def\la{\leftarrow}
\def\La{\Leftarrow}
\def\lra{\leftrightarrow}
\def\Lra{\Leftrightarrow}
\def\op{\oplus}
\def\ox{\otimes}
\def\ra{\rightarrow}
\def\Ra{\Rightarrow}
\def\su{\subset}
\def\sue{\subseteq}
\def\sm{\setminus}
\def\we{\wedge}
\newcommand{\bra}[1]{\langle#1|}
\newcommand{\ket}[1]{|#1\rangle}
\newcommand{\proj}[1]{| #1\rangle\!\langle #1 |}
\newcommand{\ketbra}[2]{|#1\rangle\!\langle#2|}
\newcommand{\braket}[2]{\langle#1|#2\rangle}
\newcommand{\wetw}[2]{|#1\rangle\wedge|#2\rangle}
\newcommand{\weth}[3]{|#1\rangle\wedge|#2\rangle\wedge|#3\rangle}
\newcommand{\wefo}[4]{|#1\rangle\wedge|#2\rangle\wedge|#3\rangle\wedge|#4\rangle}
\newcommand{\norm}[1]{\lVert#1\rVert}
\newcommand{\abs}[1]{|#1|}

\newcommand{\tbc}{\red{TO BE CONTINUED...}}

\newcommand{\opp}{\red{OPEN PROBLEM}.~}


%5=color

\newcommand{\red}{\textcolor{red}}
% open questions
\newcommand{\blue}{\textcolor{blue}}
% suspicious result or derivation
\newcommand{\green}{\textcolor{green}}
\newcommand{\white}{\textcolor{white}}

%6=journal

\newcommand{\cmp}{Comm. Math. Phys.}
\newcommand{\ieee}{IEEE. Trans. Inf. Theory}
\newcommand{\jmp}{J. Math. Phys.}
\newcommand{\jpa}{J. Phys. A}
\newcommand{\natpho}{Nat. Photon.}
\newcommand{\natphy}{Nature Phys.}
\newcommand{\natl}{Nature (London)}
\newcommand{\njp}{New. J. Phys.}
\newcommand{\phyr}{Phys. Rep.}
\newcommand{\pla}{Phys. Lett. A}
\newcommand{\sci}{Science}
% APS journals, such as jmo, pra, prl, rmp etc are defined by default.

%7=To make unique the abbreviation for the title of parts and sections, we follow the rules:

%a. Put "q" ahead of the word of quantum physics sections, e.g., %physics=qphysics;

%b. Put "c" ahead of the word of computer sections, e.g., %NPCvsNP=cNPCvsNP;

%c. Put "m" ahead of the word of mathematics sections, e.g., %matrix=mmatrix;

%d. When there are identical abbreviations, such as quantum operations, quantum operations and entanglement,
%quantum operations and distinguishing, mark them as qoperations, qoentanglement, qodistingushing respectively. In other words,
%take the first alphabet of the first n words in turn.

%8=To make unique the abbreviation for the references, we follow the rules:

%a. When there are only one name, take the last name and year, e.g., Lin Chen 2011=chen11;

%b. When there are two names take the initial alphabets of both last names and year, e.g.,
%Lin Chen and Huangjun Zhu, 2011=cz11;

%c. When there are three or more names, take the initial alphabets of both last names and year, e.g.,
%Lin Chen, Huangjun Zhu, and Tzu-Chieh Wei, 2011=czw11;

%d. When there are identical abbreviations, put the publication name in the end; e.g., hhh00PRL and hhh00PRA;

%e. When the reference is a book, put "book" in the end; e.g., harris92book.

\def\Dbar{\leavevmode\lower.6ex\hbox to 0pt
{\hskip-.23ex\accent"16\hss}D}
%% \author {{ Dragomir {\v{Z} \Dbar}okovi{\'c}}}

%% \affiliation{Department of Pure Mathematics and Institute for
%% Quantum Computing, University of Waterloo, Waterloo, Ontario, %% N2L 3G1, Canada} \email{djokovic@uwaterloo.ca}